% Options for packages loaded elsewhere
\PassOptionsToPackage{unicode}{hyperref}
\PassOptionsToPackage{hyphens}{url}
%
\documentclass[
  ignorenonframetext,
]{beamer}
\usepackage{pgfpages}
\setbeamertemplate{caption}[numbered]
\setbeamertemplate{caption label separator}{: }
\setbeamercolor{caption name}{fg=normal text.fg}
\beamertemplatenavigationsymbolsempty
% Prevent slide breaks in the middle of a paragraph
\widowpenalties 1 10000
\raggedbottom
\setbeamertemplate{part page}{
  \centering
  \begin{beamercolorbox}[sep=16pt,center]{part title}
    \usebeamerfont{part title}\insertpart\par
  \end{beamercolorbox}
}
\setbeamertemplate{section page}{
  \centering
  \begin{beamercolorbox}[sep=12pt,center]{part title}
    \usebeamerfont{section title}\insertsection\par
  \end{beamercolorbox}
}
\setbeamertemplate{subsection page}{
  \centering
  \begin{beamercolorbox}[sep=8pt,center]{part title}
    \usebeamerfont{subsection title}\insertsubsection\par
  \end{beamercolorbox}
}
\AtBeginPart{
  \frame{\partpage}
}
\AtBeginSection{
  \ifbibliography
  \else
    \frame{\sectionpage}
  \fi
}
\AtBeginSubsection{
  \frame{\subsectionpage}
}

\usepackage{amsmath,amssymb}
\usepackage{iftex}
\ifPDFTeX
  \usepackage[T1]{fontenc}
  \usepackage[utf8]{inputenc}
  \usepackage{textcomp} % provide euro and other symbols
\else % if luatex or xetex
  \usepackage{unicode-math}
  \defaultfontfeatures{Scale=MatchLowercase}
  \defaultfontfeatures[\rmfamily]{Ligatures=TeX,Scale=1}
\fi
\usepackage{lmodern}
\ifPDFTeX\else  
    % xetex/luatex font selection
\fi
% Use upquote if available, for straight quotes in verbatim environments
\IfFileExists{upquote.sty}{\usepackage{upquote}}{}
\IfFileExists{microtype.sty}{% use microtype if available
  \usepackage[]{microtype}
  \UseMicrotypeSet[protrusion]{basicmath} % disable protrusion for tt fonts
}{}
\makeatletter
\@ifundefined{KOMAClassName}{% if non-KOMA class
  \IfFileExists{parskip.sty}{%
    \usepackage{parskip}
  }{% else
    \setlength{\parindent}{0pt}
    \setlength{\parskip}{6pt plus 2pt minus 1pt}}
}{% if KOMA class
  \KOMAoptions{parskip=half}}
\makeatother
\usepackage{xcolor}
\newif\ifbibliography
\setlength{\emergencystretch}{3em} % prevent overfull lines
\setcounter{secnumdepth}{-\maxdimen} % remove section numbering


\providecommand{\tightlist}{%
  \setlength{\itemsep}{0pt}\setlength{\parskip}{0pt}}\usepackage{longtable,booktabs,array}
\usepackage{calc} % for calculating minipage widths
\usepackage{caption}
% Make caption package work with longtable
\makeatletter
\def\fnum@table{\tablename~\thetable}
\makeatother
\usepackage{graphicx}
\makeatletter
\def\maxwidth{\ifdim\Gin@nat@width>\linewidth\linewidth\else\Gin@nat@width\fi}
\def\maxheight{\ifdim\Gin@nat@height>\textheight\textheight\else\Gin@nat@height\fi}
\makeatother
% Scale images if necessary, so that they will not overflow the page
% margins by default, and it is still possible to overwrite the defaults
% using explicit options in \includegraphics[width, height, ...]{}
\setkeys{Gin}{width=\maxwidth,height=\maxheight,keepaspectratio}
% Set default figure placement to htbp
\makeatletter
\def\fps@figure{htbp}
\makeatother

\makeatletter
\@ifpackageloaded{caption}{}{\usepackage{caption}}
\AtBeginDocument{%
\ifdefined\contentsname
  \renewcommand*\contentsname{Table of contents}
\else
  \newcommand\contentsname{Table of contents}
\fi
\ifdefined\listfigurename
  \renewcommand*\listfigurename{List of Figures}
\else
  \newcommand\listfigurename{List of Figures}
\fi
\ifdefined\listtablename
  \renewcommand*\listtablename{List of Tables}
\else
  \newcommand\listtablename{List of Tables}
\fi
\ifdefined\figurename
  \renewcommand*\figurename{Figure}
\else
  \newcommand\figurename{Figure}
\fi
\ifdefined\tablename
  \renewcommand*\tablename{Table}
\else
  \newcommand\tablename{Table}
\fi
}
\@ifpackageloaded{float}{}{\usepackage{float}}
\floatstyle{ruled}
\@ifundefined{c@chapter}{\newfloat{codelisting}{h}{lop}}{\newfloat{codelisting}{h}{lop}[chapter]}
\floatname{codelisting}{Listing}
\newcommand*\listoflistings{\listof{codelisting}{List of Listings}}
\makeatother
\makeatletter
\makeatother
\makeatletter
\@ifpackageloaded{caption}{}{\usepackage{caption}}
\@ifpackageloaded{subcaption}{}{\usepackage{subcaption}}
\makeatother
\ifLuaTeX
  \usepackage{selnolig}  % disable illegal ligatures
\fi
\usepackage{bookmark}

\IfFileExists{xurl.sty}{\usepackage{xurl}}{} % add URL line breaks if available
\urlstyle{same} % disable monospaced font for URLs
\hypersetup{
  pdftitle={Untitled},
  hidelinks,
  pdfcreator={LaTeX via pandoc}}

\title{Untitled}
\author{}
\date{}

\begin{document}
\frame{\titlepage}

\begin{frame}{Choices}
\phantomsection\label{choices}
\begin{itemize}
\item
  Choice is fundamental to human existence, ranging from daily
  individual decisions to major collective choices that shape society
\item
  Individual choices encompass:

  \begin{itemize}
  \tightlist
  \item
    Routine decisions about time and money management
  \item
    Major life decisions regarding housing, family, and career paths
  \end{itemize}
\item
  Collective choices involve:

  \begin{itemize}
  \tightlist
  \item
    Creation of laws and regulations
  \item
    Management of public resources
  \item
    Decision-making through voting or representatives
  \end{itemize}
\item
  All choices involve trade-offs:

  \begin{itemize}
  \tightlist
  \item
    Selecting one option means foregoing others
  \item
    Both individual and collective choices require evaluating
    alternatives based on preferences
  \end{itemize}
\end{itemize}
\end{frame}

\begin{frame}
\begin{itemize}
\tightlist
\item
  People's choices reveal their preferences:

  \begin{itemize}
  \tightlist
  \item
    Choosing an activity shows it was ``worth'' its cost
  \item
    Observed choices help understand what people value
  \end{itemize}
\item
  Nonmarket valuation focuses on choices not captured by traditional
  markets:

  \begin{itemize}
  \tightlist
  \item
    Environmental goods and services (clean air, wilderness)
  \item
    Natural amenities that aren't directly purchased
  \item
    Public resources requiring collective decision-making
  \end{itemize}
\item
  Understanding preferences helps inform policy decisions:

  \begin{itemize}
  \tightlist
  \item
    Guides allocation of public funds
  \item
    Assists in evaluating environmental preservation
  \item
    Supports cost-benefit analysis of public projects
  \end{itemize}
\end{itemize}
\end{frame}

\begin{frame}{Market failure and choices}
\phantomsection\label{market-failure-and-choices}
\begin{itemize}
\tightlist
\item
  The ``invisible hand'' principle has limitations:

  \begin{itemize}
  \tightlist
  \item
    While individual choices can benefit society, this doesn't always
    work for environmental goods
  \item
    The principle assumes markets exist for all goods and services
    people value
  \end{itemize}
\item
  Market failures in environmental goods lead to undersupply:

  \begin{itemize}
  \tightlist
  \item
    Without markets, providers can't receive payment for environmental
    benefits
  \item
    Private landowners lack incentives to protect habitat when they
    can't monetize the benefits
  \item
    The true social value of environmental goods isn't reflected in
    market transactions
  \end{itemize}
\end{itemize}
\end{frame}

\begin{frame}
\begin{itemize}
\tightlist
\item
  Environmental externalities create economic inefficiencies:

  \begin{itemize}
  \tightlist
  \item
    Negative externalities occur when actions harm others without
    compensation
  \item
    Companies don't pay for environmental damage, leading to excessive
    pollution
  \item
    Private costs don't include environmental impacts, distorting
    decision-making
  \end{itemize}
\item
  Two solutions exist for market failures:

  \begin{itemize}
  \tightlist
  \item
    Create new markets for environmental goods where possible
  \item
    Implement government interventions through regulations or direct
    provision
  \end{itemize}
\item
  Nonmarket valuation serves crucial roles:

  \begin{itemize}
  \tightlist
  \item
    Helps quantify environmental benefits that markets don't capture
  \item
    Provides information to address market failures
  \item
    Supports policy decisions about environmental protection
  \end{itemize}
\end{itemize}
\end{frame}

\begin{frame}{Non-market Valuation}
\phantomsection\label{non-market-valuation}
\begin{itemize}
\tightlist
\item
  The evolution of nonmarket valuation spans several decades:

  \begin{itemize}
  \tightlist
  \item
    Originated in 1950s U.S. for water resource project analysis
  \item
    Gained momentum in 1980s through key federal actions
  \item
    Executive Order 12291 mandated benefit-cost analyses
  \item
    Environmental legislation required damage assessments
  \end{itemize}
\item
  Applications expanded to include:

  \begin{itemize}
  \tightlist
  \item
    Environmental regulation benefit assessment
  \item
    Natural resource damage compensation
  \item
    Land and water management decisions
  \item
    Ecosystem services valuation
  \end{itemize}
\end{itemize}
\end{frame}

\begin{frame}
\begin{itemize}
\tightlist
\item
  Growing recognition of ecosystem services drove wider adoption:

  \begin{itemize}
  \tightlist
  \item
    Non-economists began showing interest
  \item
    Environmental degradation raised awareness
  \item
    Policy decisions often overlooked ecosystem value
  \item
    Need emerged to quantify natural benefits
  \end{itemize}
\item
  Landmark ecosystem valuation studies emerged:

  \begin{itemize}
  \tightlist
  \item
    1997 study estimated global ecosystem value at \$33 trillion
  \item
    Generated controversy and critique from economists
  \item
    Sparked important discussions about valuation methods
  \item
    Highlighted need for proper methodology
  \end{itemize}
\item
  Current state of nonmarket valuation:

  \begin{itemize}
  \tightlist
  \item
    Serves critical role in environmental decision-making
  \item
    Helps quantify previously ignored natural benefits
  \item
    Requires careful understanding of proper techniques
  \item
    Continues to evolve with new applications and methods
  \end{itemize}
\end{itemize}
\end{frame}

\begin{frame}{Values vs valuation}
\phantomsection\label{values-vs-valuation}
\begin{itemize}
\tightlist
\item
  Economic valuation differentiates between two types of values:

  \begin{itemize}
  \tightlist
  \item
    Held values: Core principles like loyalty, freedom, or environmental
    stewardship
  \item
    Assigned values: Specific valuations, such as how much someone would
    pay to preserve a local forest or clean up a polluted lake
  \end{itemize}
\item
  Assigned values are influenced by multiple factors:

  \begin{itemize}
  \tightlist
  \item
    Individual perceptions: A hiker might value wilderness differently
    than a developer
  \item
    Personal held values: Someone who believes in environmental
    protection might assign higher value to endangered species
    preservation
  \item
    Context: The value assigned to clean air might be higher in a
    heavily polluted city versus a rural area
  \item
    External circumstances: Income levels affect how much people can pay
    for environmental improvements
  \end{itemize}
\end{itemize}
\end{frame}

\begin{frame}
\begin{itemize}
\tightlist
\item
  Nonmarket valuation specifically focuses on:

  \begin{itemize}
  \tightlist
  \item
    Measuring assigned values: Quantifying how much people value
    improving air quality from level A to B
  \item
    Relative changes: Comparing outcomes like having a protected wetland
    versus developing it
  \item
    Practical examples: Determining the value of preserving a national
    park or reducing water pollution in a river
  \item
    Trade-off decisions: Whether to spend \$20 million on forest
    preservation or a new museum
  \end{itemize}
\item
  Key principles of environmental valuation:

  \begin{itemize}
  \tightlist
  \item
    Values are relative: Clean air is valued against current pollution
    levels
  \item
    Specific outcomes: Preserving 1,000 acres of wetland versus abstract
    environmental protection
  \item
    Practical applications: Assessing damages from oil spills or
    benefits of emissions regulations
  \item
    Real-world choices: Deciding between expanding a highway or
    protecting adjacent wildlife habitat
  \end{itemize}
\end{itemize}
\end{frame}

\begin{frame}
\begin{itemize}
\tightlist
\item
  Economic approach to valuation:

  \begin{itemize}
  \tightlist
  \item
    Real-world applications: Evaluating compensation for environmental
    damage from chemical spills
  \item
    Policy decisions: Determining appropriate pollution control
    standards
  \item
    Project assessment: Analyzing costs and benefits of dam construction
  \item
    Resource management: Deciding optimal harvest levels for fisheries
    or forests
  \end{itemize}
\end{itemize}
\end{frame}

\begin{frame}{Concept of value}
\phantomsection\label{concept-of-value}
\begin{itemize}
\tightlist
\item
  Economic theory defines value through trade-offs:

  \begin{itemize}
  \tightlist
  \item
    Value = Maximum amount one would give up to gain something
  \item
    Example: How much money someone would sacrifice to preserve a local
    park
  \item
    Negative values exist too: Compensation required to accept pollution
  \end{itemize}
\item
  Two key principles of economic value:

  \begin{itemize}
  \tightlist
  \item
    Higher willingness to sacrifice indicates higher value
  \item
    Example: If someone would pay ₹100 to save forest A but only ₹50 for
    forest B, they value A more
  \item
    Values can be compared by measuring sacrifice amounts
  \item
    Example: Choosing between ₹20M for air quality improvement or water
    cleanup
  \end{itemize}
\end{itemize}
\end{frame}

\begin{frame}
\begin{itemize}
\tightlist
\item
  Trade-offs can be measured in different ways:

  \begin{itemize}
  \tightlist
  \item
    Monetary terms (most common): Rupees willing to pay
  \item
    Risk trade-offs: Accepting one risk to reduce another
  \item
    Example: Accepting slightly higher traffic risk to reduce flood risk
  \item
    Time trade-offs: Hours willing to volunteer for environmental
    cleanup
  \end{itemize}
\item
  Key economic value concepts:

  \begin{itemize}
  \tightlist
  \item
    Willingness to Pay (WTP): Amount someone would pay for improvement
  \item
    Example: ₹50 monthly for cleaner air
  \item
    Willingness to Accept (WTA): Compensation required to give up
    something
  \item
    Example: ₹1000 to accept loss of neighborhood green space
  \end{itemize}
\item
  Benefits in economic valuation:

  \begin{itemize}
  \tightlist
  \item
    Specifically means monetary value assigned
  \item
    Allows comparison across different projects
  \item
    Enables cost-benefit analysis
  \item
    Example: ₹5M benefit from wetland preservation versus ₹3M cost
  \end{itemize}
\end{itemize}
\end{frame}

\begin{frame}
!(){[}``s1\_valuing\_env\_goods\_files/f\_1.png''{]}
\end{frame}



\end{document}
